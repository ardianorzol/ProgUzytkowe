\documentclass[a4paper,12pt]{article}
\usepackage{algorithmic}
\usepackage[MeX]{polski}
\usepackage[utf8]{inputenc}

%opening
\title{Matematyka}
\author{Adrian Orzoł}

\begin{document}

\maketitle

\section{Wzory}
\begin{equation}\label{eq:Pierwszy}\lim_{n \to \infty}\sum_{k=1}^n \frac{1}{k^2}=\frac{\pi^2}{6}\end{equation}
\begin{equation}\label{eq:Drugi}\prod_{i=2}^{n=i^2}=\frac{\lim_{}{}^{n \to 4}(1+\frac{1}{n})^n}{\sum_{}{}k(\frac{1}{n})}\end{equation}
Łatwo równanie (\ref{eq:sum}) doprowadzić do (\ref{eq:Drugi})
\begin{equation}\label{eq:Trzeci}\int_{2}^\infty \frac{1}{\log_{2}x}dx=\frac{1}{x}\sin x=1-\cos^2 x)\end{equation}
\begin{equation}\label{eq:Czwarty}\left(\begin{array}{cccc}a_{11}&a_{12}&\ldots&a_{1K}\\a_{21}&a_{22}&\ldots&a_{2K}\\\end{array}\end{equation}
\end{document}