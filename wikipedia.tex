\documentclass[a4paper,12pt]{article}
\usepackage[MeX]{polski}
\usepackage[utf8]{inputenc}
\usepackage{graphicx}
\usepackage{amsfonts}

%opening
\title{Grzędy (województwo dolnośląskie)}
\author{}

\begin{document}

\maketitle
Grzędy (niem. Mittelkonradswaldau) --- wieś w~Polsce położona w~województwie dolnośląskim, w~powiecie wałbrzyskim, w~gminie Czarny Bór, u~podnóża Krzeszowskich Wzgórz w~Kotlinie Kamiennogórskiej oraz Pasma Lesistej i~Czarnego Lasu (Góry Kamienne) w~Sudetach Środkowych.
\begin{figure}
\begin{center}
\includegraphics[width=0.5\hsize]{wikipediaZdjecia/dobry.eps}
\caption{Grzędy od strony Ptasiego Śpiewu. W tle widoczna Śnieżka.}
\end{center}
\end{figure}

\begin{table}
\begin{tabular}{cccc}
\hline
\textbf{Rok}&\textbf{Ogółem}&\textbf{Kobiety}&\textbf{Mężczyźni}\\
\hline
2014&1000&455&545\\
2015&960&435&525\\
2016&1100&535&565\\
\end{tabular}
\caption{Liczba ludności w latach 2014--2016}
\end{table}


\tableofcontents

\section{Podział administracyjny}
W latach 1975--1998 miejscowość administracyjnie należała do województwa wałbrzyskiego.
\section{Turystyka i rekreacja}
W Grzędach znajduje się zalew rekreacyjny, położony u podnóża gór Małego i~Dużego Dzikowca. Miejscowość przecina pieszy Główny Szlak Sudecki. Przez~miejscowość przebiega również szlak rowerowy gminy Czarny Bór.
\section{Zabytki}
Według rejestru Narodowego Instytutu Dziedzictwa na listę zabytków wpisane są:
\begin{itemize}
\item kościół filialny pod wezwaniem św. Jadwigi, z XIII w., przebudowany na gotycki około 1550 r. oraz w XVII i XVIII w., remontowany w~1964 r. 
Prezbiterium nakryte jest sklepieniem krzyżowo-żebrowym, a~we wnętrzu znajduje się rzeźbiony renesansowy ołtarz z drugiej połowy XVI~w.
\item zamek ,,Wojaczów'' w ruinie, z XIV--XV wieku
\end{itemize}
inne:
\begin{itemize}
\item tujoklon, na cmentarzu obok kościoła rośnie unikat, być może w skali europejskiej, prawdziwy fenomen natury --- tuja i klon o wspólnym pniu.
\end{itemize}
\section{Zobacz też}
\begin{itemize}
\item Kościół św. Jadwigi w Grzędach
\item Grzędy
\item Grzędy Górne
\end{itemize}
\section{Przypisy}
\begin{enumerate}
\item GUS: Ludność --- struktura według ekonomicznych grup wieku. Stan w~dniu 31.03.2011 r.
\item  Rozporządzenie Ministrów: Administracji Publicznej i Ziem Odzyskanych z dnia 12 listopada 1946 r. o przywróceniu i ustaleniu urzędowych nazw miejscowości (M.P. z 1946 r. Nr 142, poz. 262)
\item Sudety Środkowe. Skala 1:40000. Wyd. 6. Jelenia Góra: Wydawnictwo Turystyczne Plan, 2011. ISBN 978-83-62917-84-6.
\item Rejestr zabytków nieruchomych woj. dolnośląskiego. Narodowy Instytut Dziedzictwa. [dostęp 20.10.2012]. s. 186.
\end{enumerate}
\end{document}