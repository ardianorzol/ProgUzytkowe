\documentclass{beamer}
\usepackage[utf8]{inputenc} 
\title{Elektrownia orbitalna}
\author{Adrian Orzoł}
\institute{UWM}
\date{\today}
\usepackage{amsfonts}
\usepackage[MeX]{polski}
\begin{document}
\frame{\titlepage}



\begin{frame}
\frametitle{Spis Treści}
\tableofcontents
\end{frame}



\section{Wstęp}
\begin{frame}{Wstęp}
Elektrownia orbitalna --- sztuczny satelita umieszczony na~orbicie, działający jak elektrownia słoneczna i przesyłający energię na~Ziemię za pomocą mikrofal do specjalnej anteny odbiorczej.
\end{frame}

\section{Historia}
\begin{frame}{Historia}
\begin{itemize}
\item Pierwsze projekty elektrowni orbitalnej pochodzą z lat sześćdziesiątych XX wieku;
\pause
\item Początkowo uznawane były za niewykonalne, z powodu braku technologii do efektywnego przesyłania energii z orbity na Ziemię;
\pause
\item Na początku XXI wieku pojawiły się pomysły odświeżenia projektu, w USA i w Japonii.
\end{itemize}
\end{frame}
\section{Opis}
\begin{frame}{Opis}
Główne elementy elektrowni orbitalnej to:
\pause
\begin{itemize}
\item kolektor, zwykle zbudowany z baterii słonecznych;
\pause
\item antena mikrofalowa, skierowana na Ziemię;
\pause
\item duża antena odbiorcza, umieszczona na powierzchni Ziemi.
\end{itemize}
\end{frame}
\begin{frame}{Opis c.d.}
Kolektor słoneczny może mieć konstrukcję podobną jak jego naziemne odpowiedniki.
\end{frame}
\section{Problemy}
\begin{frame}{Problemy}
Największymi problemamy związanymi z elektrowniami orbitalnymi są:
\end{frame}
\begin{frame}{Koszt wyniesienia na orbitę}
\begin{itemize}
\item Obecnie cena to 6--11 tys. \$ za kilogram;
\item Opłacanla cena musi być niższa o 400-500 \$ za kilogram.
\end{itemize}
\end{frame}
\begin{frame}{Koszt wyniesienia na orbitę c.d.}
Istnieje kilka pomysłów znacznego obniżenia tej wartości:
\begin{itemize}
\pause
\item Zbudowanie paneli z materiałów księżycowych;
\pause
\item Wykorzystanie planetoid;
\pause
\item Wyniesienie elektrowni za pomocą windy kosmicznej.
\end{itemize}
\end{frame}
\begin{frame}{Bezpieczeństwo}
Wykorzystanie mikrofal do przesyłania energii jest najbardziej kontrowersyjnym elementem projektu. W rzeczywistości są one bezpieczne.
\end{frame}
\section{Porównanie}
\begin{frame}{Porównanie do zwykłej elektrowni słonecznej}
\begin{itemize}
\item Elektrownia orbitalna umożliwia wielokrotnie efektywniejsze wykorzystanie ogniw słonecznych;
\pause
\item Elektrownia naziemna wymaga kilkukrotnie większego obszaru;
\pause
\item Antena odbiorcza jest przezroczysta, prosta w konstrukcji i tania;
\end{itemize}
\end{frame}
\begin{frame}{Porównanie c.d.}
\begin{itemize}
\item<1-3> Klasyczna elektrownia słoneczna nie produkuje energii nocą;
\pause
\item<2-3> Czynniki pogodowe wpływają na efektywność elektrowni naziemnych i powodują ich zużywanie;
\pause
\item<-3> Położenie naziemnej elektrowni jest ustalone.
\item<4> Dziękuję za uwagę.
\end{itemize}
\end{frame}
\end{document}
