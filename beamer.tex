\documentclass{beamer}
\usepackage[utf8]{inputenc} 
\title{Elektrownia orbitalna}
\author{Adrian Orzoł}
\institute{UWM}
\date{\today}
\usepackage{amsfonts}
\usepackage[MeX]{polski}
\begin{document}
\frame{\titlepage}



\begin{frame}
\frametitle{Spis Treści}
\tableofcontents
\end{frame}



\section{Wstęp}
\begin{frame}{Wstęp}
Elektrownia orbitalna --- proponowany sztuczny satelita umieszczony na wysokiej orbicie, działający jak elektrownia słoneczna i przesyłający energię na Ziemię za pomocą mikrofal do specjalnej anteny odbiorczej.
\end{frame}

\begin{frame}{Odległość, cd.}
\begin{itemize}
\item<2-3> W Euklidesowej przestrzeni trójwymiarowej odległość między punktami $(x_1, y_1, z_1)$ i~$(x_2, y_2, z_2)$ równa jest
$$d=\sqrt{(x_2-x_1)^2+(y_2-y_1)^2+(z_2-z_1)^2}$$
\item<-3> W ogólnej Euklidesowej  przestrzeni $\mathbb{R}^n$ odległość między $x$ i~$y$ obliczana jest według wzoru
$$d=|x-y|=\sqrt{\sum^n_{i=1}{|x_i-y_i|^2}}$$
\item<4> That's all, folks
\end{itemize}
\end{frame}
\end{document}
